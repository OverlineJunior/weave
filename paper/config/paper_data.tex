\titulo{Implementação de um Protótipo para uma Linguagem de Programação Orientada ao Entity Component System}
\autor{Francisco Sebastiany Junior}
\orientador{Me. Luciano S. Cardoso}
\data{2025}

\codigoautor{A000}
\codigopublicacao{ CDD: 000 }

\profbancaum{Dra. Alessandra Bussador}
\profbancadois{Me. Adélio de Souza Conter}
\instituicaoprofum{Centro Universitário Dinâmica das Cataratas}
\instituicaoprofdois{Centro Universitário Dinâmica das Cataratas}
\databanca{12 de Novembro de 2025}

\textoaprovacao{\imprimirtipotrabalho\ aprovado como requisito obrigatório para obtenção do título de Bacharel em \imprimircurso\ do \imprimirinstituicao, pela seguinte banca examinadora:}

\textoepigrafe{%
``I think the next century will be the century of complexity.''\\
\-- Stephen Hawking
}

\textoresumo{Este trabalho apresenta a implementação de um protótipo de interpretador para uma linguagem de programação orientada ao \textit{Entity Component System} (ECS), um padrão arquitetural amplamente utilizado em desenvolvimento de jogos, mas também aplicado em outras áreas, como \textit{frameworks} de computação gráfica e \textit{brokers} de IoT. Diferente da abordagem tradicional, onde o ECS é abstraído por meio de bibliotecas específicas, este estudo propõe tratá-lo como um conceito de primeira classe em uma linguagem de programação. O desenvolvimento foi feito em Rust, tomando a linguagem educacional Jlox como referência para a fase de análise léxica, utilizando a biblioteca Chumsky para a análise sintática e formalizando conceitualmente a fase de interpretação. O trabalho documenta as decisões de implementação, os desafios enfrentados e as limitações do protótipo, destacando sua viabilidade como base para pesquisas futuras. Os resultados demonstram que a implementação de uma linguagem orientada a ECS é possível e semelhante à de linguagens tradicionais, mas também destacam a necessidade de uma linguagem de implementação com melhor suporte à tipagem dinâmica para superar as limitações encontradas.}
\palavrachaveum{\textit{Entity Component System}}
\palavrachavedois{Interpretador}
\palavrachavetres{Linguagem de Programação}
\palavrachavequatro{Rust}

\textoabstract{This study presents the implementation of a prototype interpreter for a programming language oriented toward the Entity Component System (ECS), an architectural pattern widely used in game development but also applied in other domains, such as computer graphics frameworks and IoT brokers. Unlike the traditional approach, in which ECS is abstracted by specific libraries, this study proposes treating it as a first-class concept in a programming language. The prototype was developed in Rust, taking the educational language Jlox as a reference for the lexical analysis phase, using the Chumsky library for syntactic analysis, and conceptually formalizing the interpretation phase. This work documents the implementation decisions, the challenges faced, and the limitations of the prototype, highlighting its viability as a basis for future research. The results show that implementing an ECS-oriented language is feasible and similar to the implementation of traditional languages, while also emphasizing the need for an implementation language with better support for dynamic typing to overcome the limitations encountered.}
\keywordum{Entity Component System}
\keyworddois{Interpreter}
\keywordtres{Programming Language}
\keywordquatro{Rust}
