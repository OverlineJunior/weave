\subsection{Padrão de Arquitetura de Software}

Um padrão de arquitetura de software (do inglês, \textit{software architecture pattern}), assim como um padrão de \textit{design} de software, é uma solução reutilizável para um problema recorrente, só que dessa vez na arquitetura de software. Enquanto um padrão de \textit{design} foca em resolver problemas mais específicos no código, um padrão de arquitetura foca em resolver problemas mais amplos, como a organização de toda a aplicação e a interação entre seus diversos componentes \cite{fundsoftwarearchitecture}.

Pode-se dizer que um dos maiores exemplos de padrão arquitetural é o \textit{Model-View-Controller} (MVC), que separa a aplicação em três componentes distintos a fim de isolar as várias responsabilidades da aplicação. Porém, neste trabalho, será utilizado outro padrão arquitetural: o \textit{Entity Component System}.
