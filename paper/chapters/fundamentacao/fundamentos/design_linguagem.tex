\subsection{Fundamentos do \textit{Design} de Linguagens de Programação}
\label{sec:design_linguagem}

Com base em \cite{conceptsoflanguages}, pode-se dizer que o \textit{design} de uma linguagem de programação é a definição da estrutura, do comportamento e dos propósitos da linguagem com base em um conjunto de princípios.

Escolhas de \textit{design} exigem consideração cuidadosa, pois geralmente exigem compromissos entre características desejáveis \cite{conceptsoflanguages}. Um exemplo é a escolha entre uma sintaxe concisa, que pode acelerar a escrita, e uma sintaxe explícita, que pode facilitar a leitura e manutenção do código.

De modo a melhor categorizar o \textit{design} de linguagens de programação, \citeonline{designconceptsinlanguages} propõe a divisão em três categorias principais: sintaxe, semântica e pragmática. A seguir, cada uma delas é descrita de forma resumida com base no mesmo autor:

\begin{itemize}
	\item \textbf{Sintaxe}: a estrutura textual da linguagem, incluindo a definição de suas palavras reconhecidas e a forma como elas podem ser combinadas para formar frases válidas;

	\item \textbf{Semântica}: o significado das frases válidas da linguagem, ou seja, o que elas representam e como são interpretadas pelo sistema;

	\item \textbf{Pragmática}: a implementação interna da linguagem, de tal forma que a semântica não se altere.
\end{itemize}

Essa divisão contribui na separação de preocupações, permitindo que a sintaxe seja definida independentemente da semântica e da pragmática, por exemplo. No caso deste trabalho, a ênfase está na sintaxe e semântica, com a pragmática sendo considerada apenas quando necessário.

\subsubsection{Características do \textit{Design} de Linguagens de Programação}

Todo \textit{design} para uma determinada linguagem de programação busca atender a certos critérios, e para isso, é necessário considerar algumas características importantes \cite{conceptsoflanguages}. Com base nisso, a \autoref{tab:criterios_caracteristicas} ilustra algumas dessas características e quais critérios elas atendem.

\begin{quadro}[h]
	\centering
	\caption{Critérios do \textit{design} de linguagens de programação e as características que os atendem.}
	{
		\begin{tabular}{|l|c|c|c|}
		\hline
		& \multicolumn{3}{c|}{\textbf{Critérios}} \\
		\cline{2-4}
		\textbf{Característica} & \textbf{Legibilidade} & \textbf{Facilidade de Escrita} & \textbf{Confiabilidade} \\ \hline
		Simplicidade & x & x & x \\ \hline
		Ortogonalidade & x & x & x \\ \hline
		Tipos de Dados & x & x & x \\ \hline
		\textit{Design} da Sintaxe & x & x & x \\ \hline
		Suporte a Abstração & & x & x \\ \hline
		Expressividade & & x & x \\ \hline
		Checagem de Tipos & & & x \\ \hline
		Tratamento de Exceções & & & x \\ \hline
		Restrição de \textit{Aliasing} & & & x \\ \hline
		\end{tabular}
	}
	\fonte{Adaptado de \citeonline{conceptsoflanguages}.}
	\label{tab:criterios_caracteristicas}
\end{quadro}

A seguir, os critérios e características ilustrados na \autoref{tab:criterios_caracteristicas} serão descritos de forma resumida com base em \citeonline{conceptsoflanguages}:

\subsubsubsection{Legibilidade}

A legibilidade de uma linguagem de programação é a facilidade com que um programador pode ler e entender o código escrito nessa linguagem. É um critério extremamente importante, pois maior parte do tempo de desenvolvimento é gasto na leitura e compreensão do código, e não na escrita \cite{howdevsspendtime}.

As principais características que afetam a legibilidade de uma linguagem de programação são as seguintes:

\begin{itemize}
	\item \textbf{Simplicidade}: inversamente proporcional ao número de construtos e regras na linguagem e proporcional à familiaridade da linguagem (construtos e regras semelhantes a outras linguagens). Múltiplos meios de expressar a mesma intenção também afetam negativamente a simplicidade;
	\item \textbf{Ortogonalidade}: um conjunto pequeno de regras que podem ser combinadas para formar construções mais complexas. Isso reduz o número de conceitos que um programador precisa entender, contribuindo para a legibilidade;
	\item \textbf{Tipos de Dados}: tipos de dados mais específicos podem tornar o código mais explícito, tornando-o mais específico à intenção do programador. Um exemplo é o uso de booleanos em vez de inteiros para representar valores verdadeiros e falsos;
	\item \textbf{Design da Sintaxe}: os itens da sintaxe, como palavras-chave, operadores, e como eles podem ser combinados, afetam a legibilidade. Idealmente, o programador deve ser capaz de entender a semântica de cada item da sintaxe sem conhecimento prévio da linguagem.
\end{itemize}

\subsubsubsection{Facilidade de Escrita}

A facilidade de escrita de uma linguagem de programação é a capacidade do programador de expressar suas intenções de forma clara e concisa. É um critério importante, pois uma linguagem que permite escrever código facilmente pode aumentar a produtividade do programador.

As principais características que afetam a facilidade de escrita de uma linguagem de programação são as seguintes:

\begin{itemize}
	\item \textbf{Simplicidade e Ortogonalidade}: menor quantidade de construtos, além de regras simples e combináveis, resultam em menos informação para o programador ter em mente, facilitando a escrita do código;
	\item \textbf{Suporte a Abstração}: sabendo que abstração é o processo de ocultar detalhes desnecessários para focar em aspectos relevantes, um bom suporte a abstração na linguagem pode permitir que o programador as crie com maior capacidade e menor esforço, facilitando a escrita do código;
	\item \textbf{Expressividade}: a capacidade de expressar intenções de forma clara e concisa. Linguagens com alta expressividade permitem que o programador escreva menos código para alcançar o mesmo resultado, o que é muitas vezes associado a açúcar sintático\footnote{Açúcar sintático (do inglês, \textit{syntactic sugar}) é um termo, originalmente criado em \citeonline{mechanicalevaluationexpressions}, usado para descrever construções de uma determinada linguagem de programação que tornam funcionalidades já existentes mais fáceis ou agradáveis de usar. Um exemplo é o laço \textit{for}, que pode tornar o código mais conciso e legível em comparação com um laço \textit{while} tradicional.}.
\end{itemize}

\subsubsubsection{Confiabilidade}

A confiabilidade de uma linguagem de programação é a capacidade de garantir que o código escrito funcione como esperado, sem erros ou comportamentos inesperados. É outro critério muito importante, pois um código confiável reduz a probabilidade de falhas e aumenta a confiança do programador na linguagem.

As principais características que afetam a confiabilidade de uma linguagem de programação são as seguintes:

\begin{itemize}
	\item \textbf{Checagem de Tipos}: é o ato de verificar se um determinado valor é do tipo esperado antes de ser usado. Isso pode prevenir erros em tempo de execução, aumentando a confiabilidade do código. A checagem de tipos pode ser feita em tempo de execução ou, preferencialmente, em tempo de compilação;
	\item \textbf{Tratamento de Exceções}: conceito que permite o tratamento de erros de forma controlada, evitando que o programa falhe inesperadamente. Algumas linguagens tornam o tratamento de exceções obrigatório, o que pode aumentar mais ainda a confiabilidade do código em troca de maior complexidade;
	\item \textbf{Restrição de \textit{Aliasing}}: \textit{aliasing} é o ato de definir dois ou mais nomes para o mesmo local de memória. Restringir o uso de \textit{aliasing} pode aumentar a confiabilidade do código, pois reduz a possibilidade de efeitos colaterais indesejados, onde uma alteração em um nome afeta outro nome que deveria ser independente;
	\item \textbf{Legibilidade e Facilidade de Escrita}: tanto a legibilidade quanto a facilidade de escrita contribuem para a confiabilidade do código, pois um código mais claro e fácil de escrever tende a ter menos erros.
\end{itemize}

% \subsubsection{Definição Formal de Sintaxe}

% BNF e EBNF.
