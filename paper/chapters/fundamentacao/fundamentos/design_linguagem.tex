% 1. Porque se preocupar com design de linguagens de programação?
% 2. Sintaxe, semântica e pragmática
% 3. Características

\subsection{Fundamentos do \textit{Design} de Linguagens de Programação}

Com base em <FONTE>, pode-se dizer que o \textit{design} de uma linguagem de programação é a definição da forma, do comportamento e dos propósitos da linguagem com base em um conjunto de princípios.

Escolhas de \textit{design} exigem consideração cuidadosa, pois geralmente exigem compromissos entre características desejáveis <FONTE>. Um exemplo é a escolha entre uma sintaxe concisa, que pode acelerar a escrita, e uma sintaxe explícita, que pode facilitar a leitura e manutenção do código.

De modo a melhor categorizar o \textit{design} de linguagens de programação, <FONTE> propõe a divisão em três categorias principais: sintaxe, semântica e pragmática. A seguir, cada uma delas é descrita de forma resumida com base em <FONTE>:

- Sintaxe:

- Semântica:

- Pragmática:
