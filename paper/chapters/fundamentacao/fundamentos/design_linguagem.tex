\subsection{Fundamentos do \textit{Design} de Linguagens de Programação}

Com base em \cite{conceptsoflanguages}, pode-se dizer que o \textit{design} de uma linguagem de programação é a definição da estrutura, do comportamento e dos propósitos da linguagem com base em um conjunto de princípios.

Escolhas de \textit{design} exigem consideração cuidadosa, pois geralmente exigem compromissos entre características desejáveis \cite{conceptsoflanguages}. Um exemplo é a escolha entre uma sintaxe concisa, que pode acelerar a escrita, e uma sintaxe explícita, que pode facilitar a leitura e manutenção do código.

De modo a melhor categorizar o \textit{design} de linguagens de programação, \citeonline{designconceptsinlanguages} propõe a divisão em três categorias principais: sintaxe, semântica e pragmática. A seguir, cada uma delas é descrita de forma resumida com base no mesmo autor:

\begin{itemize}
	\item Sintaxe: a estrutura textual da linguagem, incluindo a definição de suas palavras reconhecidas e a forma como elas podem ser combinadas para formar frases válidas;

	\item Semântica: o significado das frases válidas da linguagem, ou seja, o que elas representam e como são interpretadas pelo sistema;

	\item Pragmática: a implementação interna da linguagem, de tal forma que a semântica não se altere.
\end{itemize}

Essa divisão contribui na separação de preocupações, permitindo que a sintaxe seja definida independentemente da semântica e da pragmática, por exemplo. No caso deste trabalho, a ênfase está na sintaxe e semântica, com a pragmática sendo considerada apenas quando necessário.

\subsubsection{Características do \textit{Design} de Linguagens de Programação}

Todo \textit{design} para uma determinada linguagem de programação busca atender a certos critérios, e para isso, é necessário considerar algumas características importantes \cite{conceptsoflanguages}. Com base nisso, a \autoref{tab:criterios_caracteristicas} ilustra algumas dessas características e quais critérios elas atendem.

\begin{table}[h]
	\centering
	\caption{Critérios do \textit{design} de linguagens de programação e as características que os atendem.}
	{
		\begin{tabular}{lccc}
		\hline
		& \multicolumn{3}{c}{\textbf{Critérios}} \\
		\cline{2-4}
		\textbf{Característica} & \textbf{Legibilidade} & \textbf{Facilidade de Escrita} & \textbf{Confiabilidade} \\
		Simplicidade & x & x & x \\
		Ortogonalidade & x & x & x \\
		Tipos de Dados & x & x & x \\
		\textit{Design} da Sintaxe & x & x & x \\
		Suporte a Abstração & & x & x \\
		Expressividade & & x & x \\
		Checagem de Tipos & & & x \\
		Tratamento de Exceções & & & x \\
		Restrição de \textit{Aliasing} & & & x \\
		\hline
		\end{tabular}
	}
	\fonte{Adaptado de \citeonline{conceptsoflanguages}.}
	\label{tab:criterios_caracteristicas}
\end{table}

A seguir, as características ilustradas na \autoref{tab:criterios_caracteristicas} serão descritas de forma resumida com base em \citeonline{conceptsoflanguages}:

\subsubsubsection{Legibilidade}

\begin{itemize}
	\item a
\end{itemize}

\subsubsubsection{Facilidade de Escrita}

\begin{itemize}
	\item a
\end{itemize}

\subsubsubsection{Confiabilidade}

\begin{itemize}
	\item a
\end{itemize}

\subsubsection{Definição Formal de Sintaxe}

BNF e EBNF.
