\chapter{Metodologia}\label{ch:metodologia}

Este capítulo apresenta os procedimentos e métodos adotados para a realização do trabalho. Aqui serão detalhados o tipo e a abordagem da pesquisa, o contexto no qual o estudo se encontra, a população e a amostra analisadas, e por fim as técnicas de coleta e análise de dados utilizadas.

\section{Tipo e Abordagem de Pesquisa}

Este trabalho consiste de uma pesquisa aplicada, de natureza exploratória, cujo objetivo é investigar a implementação de uma linguagem de programação orientada a ECS como base para trabalhos futuros, visto que esse é um tema muito pouco explorado na literatura acadêmica.

A abordagem adotada é totalmente qualitativa, buscando investigar certos aspectos na implementação do protótipo da linguagem, como usabilidade e modularidade. Quando relevante, também serão feitas análises comparativas entre a implementação da linguagem proposta e a da linguagem Jlox, apresentada em \citeonline{craftinginterpreters}, que serviu como base para a implementação do protótipo.

\section{Contexto}

O trabalho se encontra no contexto dado pela intersecção da área de desenvolvimento de linguagens de programação e padrão ECS. O ECS guia o \textit{design} e a implementação da linguagem de programação proposta de forma a tornar o uso do padrão mais natural.

\section{Delimitação do Estudo}

Com a finalidade de manter o foco do trabalho em seus objetivos centrais, serão estabelecidas algumas delimitações listadas a seguir:

\begin{itemize}
    \item Não serão abordados aspectos relacionados ao desempenho da linguagem, como tempo de execução e uso de memória, devido a complexidade adicional envolvida na análise desses aspectos;
    \item Não serão abordados aspectos relacionados ao \textit{design} da linguagem, como sua sintaxe. Esses aspectos estarão reservados para a análise da implementação do protótipo;
    \item Não serão abordados aspectos relacionados ao ecossistema da linguagem proposta, como bibliotecas padrão e ferramentas de desenvolvimento;
    \item A fase de interpretação da linguagem será abordada de forma teórica, sem sua implementação. Essa decisão foi tomada devido a limitação de tempo disponível para implementar a fase.
\end{itemize}

\section{População e Amostra}

A população de referência para o estudo são linguagens de programação tradicionais e bibliotecas que implementam ou dão suporte ao padrão ECS. A amostra escolhida inclui a linguagem Rust e as bibliotecas de ECS Flecs e Bevy, por serem extremamente relevantes no contexto de ECS e por serem inovadoras em suas abordagens.

\section{Técnicas de Coleta de Dados}

Para a coleta de dados, foram utilizadas as técnicas de pesquisa bibliográfica, documental e experimental, conforme descrito a seguir:

\begin{itemize}
    \item \textbf{Pesquisa Bibliográfica}: análise de livros, artigos e publicações acadêmicas sobre linguagens de programação, ECS e \textit{design} de software;
    \item \textbf{Pesquisa Documental}: análise de documentação, repositórios de código-fonte e exemplos de uso de linguagens e bibliotecas que implementam o padrão ECS;
    \item \textbf{Pesquisa Experimental}: implementação da linguagem proposta de forma iterativa com base na análise qualitativa dos resultados.
\end{itemize}

\section{Técnicas de Análise de Dados}

A análise dos dados coletados será realizada de forma qualitativa, por meio das seguintes técnicas de análise:

\begin{itemize}
    \item \textbf{Análise Documental e Bibliográfica}: análise e interpretação dos conceitos, padrões e práticas encontrados nas fontes pesquisadas.
    \item \textbf{Análise Comparativa}: avaliação das características da implementação do protótipo em comparação com a linguagem Jlox apresentada em \citeonline{craftinginterpreters}, destacando pontos fortes e limitações.
\end{itemize}

\section{Procedimentos Metodológicos}

O procedimento metodológico adotado para o desenvolvimento deste trabalho seguiu as etapas descritas a seguir:

\begin{enumerate}
    \item \textbf{Visão Geral da Solução}: resumo de como a solução proposta será organizada e quais são seus principais componentes;
    \item \textbf{Planejamento}: definição das etapas do desenvolvimento, sintaxe da linguagem e ferramentas a serem utilizadas;
    \item \textbf{Implementação}: desenvolvimento do protótipo de interpretador, organizado com base nas fases de análise léxica, sintática e de interpretação:
          \begin{enumerate}
              \item \textbf{Análise Léxica}: implementação do \textit{lexer} responsável por converter o código-fonte em \textit{tokens};
              \item \textbf{Análise Sintática}: implementação do \textit{parser} responsável por construir a árvore de sintaxe abstrata (AST) a partir dos tokens gerados na fase anterior;
              \item \textbf{Interpretação}: definição teórica da fase de interpretação, discutindo como a AST pode ser processada para ligar o interpretador à biblioteca de ECS escolhida.
          \end{enumerate}
\end{enumerate}