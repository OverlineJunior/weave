\chapter{Metodologia}\label{ch:metodologia}

Este capítulo apresenta os procedimentos e métodos adotados para a realização do trabalho. Aqui serão detalhados o tipo e a abordagem da pesquisa, o contexto no qual o estudo se encontra, a população e a amostra analisadas, e por fim as técnicas de coleta e análise de dados utilizadas.

\section{Tipo e Abordagem de Pesquisa}

Este trabalho consiste de uma pesquisa aplicada, de natureza exploratória, cujo objetivo é investigar a implementação de uma linguagem de programação orientada a ECS como base para trabalhos futuros, visto que esse é um tema muito pouco explorado na literatura acadêmica.

A abordagem adotada é totalmente qualitativa, buscando investigar certos aspectos na implementação do protótipo da linguagem, como usabilidade e modularidade. Adicionalmente, são feitas análises comparativas com a implementação de linguagens tradicionais.

\section{Contexto}

O trabalho se encontra no contexto dado pela intersecção da área de desenvolvimento de linguagens de programação e padrão ECS. O ECS guia o \textit{design} e a implementação da linguagem de programação proposta de forma a tornar o uso do padrão mais natural.

\section{Delimitação do Estudo}

Com a finalidade de manter o foco do trabalho em seus objetivos centrais, serão estabelecidas algumas delimitações listadas a seguir:

\begin{itemize}
    \item Não serão abordados aspectos relacionados ao desempenho da linguagem, como tempo de execução e uso de memória, devido a complexidade adicional envolvida na análise desses aspectos;
    \item Não serão abordados aspectos relacionados ao \textit{design} da linguagem, como sua sintaxe. Esses aspectos estarão reservados para a análise da implementação do protótipo;
    \item Não serão abordados aspectos relacionados aoecossistema da linguagem proposta, como bibliotecas padrão e ferramentas de desenvolvimento;
    \item A fase de interpretação da linguagem será abordada de forma teórica, sem sua implementação. Essa decisão foi tomada devido a limitação de tempo disponível para implementar a fase.
\end{itemize}

\section{População e Amostra}

A população de referência para o estudo são linguagens de programação tradicionais e bibliotecas que implementam ou dão suporte ao padrão ECS. A amostra escolhida inclui a linguagem Rust e as bibliotecas de ECS Flecs e Bevy, por serem extremamente relevantes no contexto de ECS e por serem inovadoras em suas abordagens.

\section{Técnicas de Coleta de Dados}

Para a coleta de dados, foram utilizadas as técnicas de pesquisa bibliográfica, documental e experimental, conforme descrito a seguir:

\begin{itemize}
    \item \textbf{Pesquisa Bibliográfica}: análise de livros, artigos e publicações acadêmicas sobre linguagens de programação, ECS e \textit{design} de software;
    \item \textbf{Pesquisa Eocumental}: análise de documentação, repositórios de código-fonte e exemplos de uso de linguagens e bibliotecas que implementam o padrão ECS;
    \item \textbf{Pesquisa Experimental}: implementação da linguagem proposta de forma iterativa com base na análise qualitativa dos resultados.
\end{itemize}

\section{Técnicas de Análise de Dados}

A análise dos dados coletados será realizada de forma qualitativa, por meio das seguintes técnicas de análise:

\begin{itemize}
    \item \textbf{Análise Documental e Bibliográfica}: análise e interpretação dos conceitos, padrões e práticas encontrados nas fontes pesquisadas.
    \item \textbf{Análise Comparativa}: avaliação das características da implementação do protótipo em comparação com as linguagens de programação e bibliotecas selecionadas na amostra, destacando pontos fortes e limitações.
\end{itemize}

\section{Procedimentos Metodológicos}

O procedimento metodológico adotado para o desenvolvimento deste trabalho será dividido em etapas auto-contidas, ou seja, cada etapa será realizada de forma independente, mas todas interligadas para alcançar o objetivo final do trabalho. As etapas são as seguintes:

\begin{enumerate}
    \item \textbf{MVP Fundacional da Linguagem}: inclui os conceitos de entidade, componente e sistema, representando um mínimo produto viável (do inglês, \textit{minimum viable product} — MVP). A conclusão desta etapa não segere uma linguagem pronta para uso prático, mas sim um protótipo funcional que pode ser utilizado para provar a viabilidade da fundação do trabalho;
    \item \textbf{Operações Envolvendo Entidades}: inclui as operações essenciais da linguagem, como a criação e remoção de entidades, além da adição e remoção de componentes.
    \item \textbf{Inserção de Componentes em Tempo de Compilação}: inclui a operação de inserção de componentes em entidades durante o tempo de compilação. Esta etapa é essencial para a etapa seguinte, que fará uso da operação implementada nesta;
    \item \textbf{Agendamento de Sistemas}: inclui o agendamento de sistemas, permitindo que o programador defina a ordem de execução deles. Esta etapa fará uso da operação de inserção de componentes em tempo de compilação implementada na etapa anterior;
    \item \textbf{\textit{Querying} de Entidades}: inclui a operação de \textit{querying} de entidades dentro de sistemas, permitindo que o programador consulte entidades com base em seus componentes. Esta é a etapa final do trabalho, onde todos os conceitos terão sido implementados e analisados.
\end{enumerate}

Todas as etapas do trabalho serão compostas de três subetapas cronológicas: planejamento, implementação e análise, respectivamente. Elas são descritas a seguir:

\begin{enumerate}
    \item \textbf{Planejamento}: consiste na definição dos objetivos e \textit{design} dos conceitos relacionados a etapa, incluindo a definição de sintaxe e semântica;
    \item \textbf{Implementação}: consiste na implementação dos conceitos relacionados a etapa, sempre levando em consideração a linguagem como um todo;
    \item \textbf{Análise}: consiste na análise qualitativa do \textit{design} definido usando as características e critérios definidos na \autoref{sec:design_linguagem}, além de uma análise comparativa com as bibliotecas de ECS selecionadas na amostra. O processo de implementação também será analisado, buscando documentar possíveis problemas e soluções encontradas.
\end{enumerate}

A motivação por trás dessa escolha de metodologia é dada pela possibilidade de gerar um MVP logo na primeira etapa do trabalho, gerando \textit{feedback} fundacional para o restante do trabalho. Além disso, a abordagem iterativa permite uma evolução incremental, permitindo ajustes e melhorias ao longo do desenvolvimento.
