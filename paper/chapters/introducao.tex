\chapter{Introdução} \label{ch:introducao}

O \textit{Entity Component System} (ECS) é um padrão arquitetural muito adotado em desenvolvimento de jogos, uma área que exige flexibilidade para criar mecânicas complexas e modularidade para implementar novas funcionalidades com agilidade — características centrais do padrão \cite{ecsfaq}. Embora sua popularidade tenha surgido nos jogos, a organização orientada a dados e o desacoplamento entre entidades, componentes e sistemas têm se mostrado úteis também em simulações e outros contextos que demandam rápida iteração. % TODO! Usar pelo menos 2 exemplos de projetos não relacionados a jogos que usam ECS e citar.

Devido à complexidade de implementar o padrão ECS internamente \cite{ecsstorageinpics}, ele é frequentemente abstraído por meio de bibliotecas para linguagens específicas, como é o caso das bibliotecas Bevy e Flecs, dos ecossistemas Rust e C++, respectivamente \cite{bevy, flecs}. No entanto, essa abordagem pode impor limitações à expressividade do desenvolvedor, uma vez que a linguagem subjacente pode não fornecer construções ideais para representar os conceitos do ECS naturalmente.

Dada essa limitação, este trabalho propõe abstrair o ECS de uma forma diferente: por meio da implementação de um protótipo de interpretador para uma linguagem de programação que trata o ECS como um conceito de primeira classe. Com essa abordagem, é possível moldar a linguagem para atender aos requisitos específicos do padrão ECS, permitindo a exploração de abstrações que seriam difíceis ou impossíveis de serem implementadas em bibliotecas tradicionais.

Porém, devido à complexidade de criar tanto o \textit{design} quanto a implementação de um interpretador para uma linguagem de programação, este trabalho se limita ao último. Assim, o objetivo principal deste trabalho é explorar a implementação de um protótipo de interpretador funcional para uma linguagem de programação orientada a ECS, registrando as decisões tomadas e os desafios encontrados ao longo do processo.

\section{Pergunta Problema}

Como projetar e desenvolver um protótipo de interpretador para uma linguagem de programação orientada a ECS, aproveitando suas vantagens nativas para uma melhor abstração do padrão?

\section{Objetivos} \label{sec:objetivos}

\subsection{Objetivo Geral} \label{sec:obj_geral}

Investigar como o \textit{design} e a implementação de um protótipo de interpretador para uma linguagem de programação orientada a ECS podem ser efetuados.

\subsection{Objetivos Específicos}

\begin{itemize}
    \item Implementar um protótipo de interpretador funcional para tornar o padrão ECS primeira classe;
    \item Levantar decisões de \textit{design} e implementação e desafios encontrados;
    \item Determinar a viabilidade da implementação e suas limitações.
\end{itemize}

\section{Justificativa} \label{sec:justificativa}

Com o crescimento da adoção de metodologias ágeis pelas empresas, vem a demanda por arquiteturas que promovam flexibilidade no desenvolvimento de software \cite{flightdynamics}. O padrão ECS tem se destacado por atender tais demandas, especialmente nas áreas de desenvolvimento de jogos e simulações.

A maioria das abstrações feitas sobre ECS estão no formato de bibliotecas específicas para determinadas linguagens de programação, limitando a expressividade do desenvolvedor no processo de abstração.

Dado isso, este trabalho propõe uma rota alternativa: a criação do \textit{design} e implementação de uma linguagem de programação orientada a ECS. Com a capacidade de moldar a linguagem diante dos requisitos específicos do padrão ECS, essa rota propõe investigar e implementar abstrações que são difíceis ou até mesmo impossíveis de serem aplicadas em bibliotecas.
