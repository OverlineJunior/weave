\chapter{Introdução} \label{ch:introducao}

O \textit{Entity Component System} (ECS) é um padrão arquitetural muito adotado em desenvolvimento de jogos, uma área que exige flexibilidade para criar mecânicas complexas e modularidade para implementar novas funcionalidades com agilidade — características centrais do padrão \cite{ecsfaq}. Embora sua popularidade tenha surgido nos jogos, a organização orientada a dados e o desacoplamento entre entidades, componentes e sistemas também têm se mostrado úteis em outras áreas, como \textit{brokers} de IoT e \textit{frameworks} de computação gráfica \cite{ecsiotbrokers, ecscomputergraphics}.

Devido à complexidade de implementar o padrão ECS internamente \cite{ecsstorageinpics}, ele é frequentemente abstraído por meio de bibliotecas para linguagens específicas, como é o caso das bibliotecas Bevy e Flecs, dos ecossistemas Rust e C++, respectivamente \cite{bevy, flecs}. No entanto, essa abordagem pode impor limitações à expressividade do desenvolvedor, uma vez que a linguagem subjacente pode não fornecer construções ideais para representar os conceitos do ECS naturalmente.

Dada essa limitação, este trabalho propõe abstrair o ECS de uma forma diferente: por meio da implementação de um protótipo de interpretador para uma linguagem de programação que trata o ECS como um conceito de primeira classe. Com essa abordagem, é possível moldar a linguagem para atender aos requisitos específicos do padrão ECS, permitindo a exploração de abstrações que seriam difíceis ou impossíveis de serem implementadas em bibliotecas tradicionais.

Porém, devido à complexidade de criar tanto o \textit{design} quanto a implementação de um interpretador para uma linguagem de programação, este trabalho se limita ao último. Assim, o objetivo principal deste trabalho é explorar a implementação de um protótipo de interpretador funcional para uma linguagem de programação orientada a ECS, registrando as decisões tomadas e os desafios encontrados ao longo do processo.

\section{Pergunta Problema}

Como desenvolver um protótipo de interpretador para uma linguagem de programação orientada ao ECS — com as fases de análise léxica e sintática implementadas e a fase de interpretação especificada conceitualmente — registrando e avaliando as decisões tomadas?

\section{Objetivos} \label{sec:objetivos}

\subsection{Objetivo Geral} \label{sec:obj_geral}

Projetar e documentar um protótipo de interpretador para uma linguagem de programação orientada ao ECS que possa servir de base para trabalhos futuros.

\subsection{Objetivos Específicos}

\begin{itemize}
    \item Implementar as fases de análise léxica e sintática do interpretador a fim de tornar o ECS primeira classe na linguagem;
    \item Especificar conceitualmente a fase de interpretação do interpretador (mapeamento AST -> Flecs);
    \item Levantar decisões de implementação e desafios encontrados para auxiliar trabalhos futuros;
    \item Identificar limitações do protótipo e propor oportunidades de continuidade.
\end{itemize}

\section{Justificativa} \label{sec:justificativa}

Com a alta taxa de adoção de metodologias ágeis pelas empresas modernas, vem a demanda por arquiteturas que permitam alta iterabilidade, permitindo que novas funcionalidades sejam implementadas rapidamente e com baixo custo \cite{17thstateofagile, agile}. O padrão ECS se destaca nesse cenário por sua modularidade e flexibilidade, características que facilitam a adaptação e evolução de sistemas complexos \cite{ecsfaq}.

É extremamente comum que o padrão seja abstraído por meio de bibliotecas para linguagens específicas — por exemplo, só o ecossistema Rust possui pelo menos 16 bibliotecas que implementam o ECS \cite{arewegameyet}. Porém, a ideia de abstrair o ECS por meio de uma linguagem de programação é muito pouco explorada, principalmente de forma acadêmica.

Sabendo que o padrão ECS tem o seu lugar no mercado e que há uma lacuna na exploração de abstrações do padrão por meio de linguagens de programação, este trabalho propõe investigar a implementação de um protótipo de interpretador para uma linguagem orientada ao ECS.

Com o protótipo finalizado, espera-se que este trabalho sirva como base para futuras pesquisas e desenvolvimentos na área, incentivando a exploração de novas formas de abstração do padrão ECS por meio de linguagens de programação.
