\subsection{Padrão de Design de Software}

Um padrão de design de software (do inglês, \textit{software design pattern}) é uma solução reutilizável para um determinado problema recorrente no design de software. Esses padrões são descrições gerais de como resolver tais problemas, e não implementações específicas.

De acordo com o site didático Refactoring Guru de \citeauthor{refactoringguru}, os padrões podem ser classificados em três categorias:

\begin{itemize}
    \item Padrões Criacionais: tratam da criação de objetos, focando em como instanciá-los de forma a resolver problemas específicos. Abrange padrões como \textit{Singleton}, \textit{Builder} e \textit{Prototype};
    \item Padrões Estruturais: tratam da composição de objetos em uma estrutura maior, além de como eles interagem entre si. Abrange padrões como \textit{Adapter}, \textit{Decorator} e \textit{Facade};
    \item Padrões Comportamentais: tratam de algoritmos e a atribuição de responsabilidades entre objetos. Abrange padrões como \textit{Observer}, \textit{Strategy} e \textit{Visitor}.
\end{itemize}

É importante notar como o autor explica os vários padrões usando o paradigma de programação orientado a objetos — por mais que esse seja o caso, nem todos os padrões de design são exclusivos a esse paradigma. Um exemplo é o padrão \textit{Observer}, cujo conceito é de grande importância para o paradigma de programação reativa (do inglês, \textit{reactive programming}) \cite{reactivex}.
