\subsection{Biblioteca Chumsky}

Chumsky é uma biblioteca de análise sintática para Rust. Ela é baseada no conceito de \textit{parser combinators}\footnote{Um \textit{parser combinator} consiste na combinação de parsers mais simples para criar parsers mais complexos, assim como é de costume compor uma função maior de funções menores.}, e permite que a definição de \textit{parsers} seja feita de forma declarativa. Seu escopo abrange tanto gramáticas livres de contexto quanto gramáticas sensíveis ao contexto.

Ao usar a biblioteca para construir um \textit{parser}, nota-se a influência do paradigma funcional. Por mais que seja um paradigma mais incomum, seu uso na biblioteca torna o processo de construção do \textit{parser} muito parecido com a construção de uma gramática formal, como demonstrado pelos comentários no \autoref{cod:chumsky_example}.

\codigoRust
\lstinputlisting[
	language=Rust,
	label=cod:chumsky_example,
	caption=\textit{Parser} para uma gramática de expressões aritméticas simples usando a biblioteca Chumsky.
]{../codes/chumsky_example.rs}
\vspace{-1em}
\fonte{Adaptado de \textcite{logos}.}

De acordo com a classificação do teste de desempenho da biblioteca e seus competidores localizada na \autoref{tab:chumsky_benchmark}, Chusmky tem a capacidade de ser a biblioteca de análise sintática mais rápida para Rust.

\FloatBarrier

\begin{table}[H]
	\centering
	\caption{Classificação do teste de desempenho da biblioteca Chumsky e competidores.}
	{
		\begin{tabular}{lll}
			\hline
			\textbf{Classificação} & \textbf{Biblioteca}      & \textbf{Tempo de Execução} \\ \hline
			1                      & chumsky (check-only) & 140.77 µs                  \\
			2                      & winnow               & 178.91 µs                  \\
			3                      & chusmky              & 210.43 µs                  \\
			4                      & sn                   & 237.94 µs                  \\
			5                      & serde\_json          & 477.41 µs                  \\
			6                      & nom                  & 526.52 µs                  \\
			7                      & pest                 & 1.9706 ms                  \\
			8                      & pom                  & 13.730 ms                  \\ \hline
		\end{tabular}
	}
	\fonte{Adaptado de \textcite{chumsky}.}
	\label{tab:chumsky_benchmark}
\end{table}

Por fim, o uso da biblioteca Chumsky estará na implementação da análise sintática, e será utilizada em conjunto com a biblioteca Logos na implementação do interpretador como um todo. Assim como foi o caso com Logos, a escolha de Chumsky se deve à sua maturidade.
