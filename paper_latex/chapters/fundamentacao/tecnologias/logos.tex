\subsection{Biblioteca Logos}

Logos é uma biblioteca de análise léxica para Rust. Ela consiste na definição de \textit{tokens} através de \textit{macros} e expressões regulares, tornando o código extremamente conciso.

O \autoref{cod:logos_example} ilustra a definição dos \textit{tokens} de uma calculadora simples, onde cada \textit{token} é definido através de uma expressão regular. A biblioteca automaticamente gera o analisador léxico, dado por \texttt{Token::lexer}, que pode ser usado para analisar uma string e retornar os \textit{tokens} correspondentes.

\lstinputlisting[
	language=Rust,
	label=cod:logos_example,
	caption=Análise léxica para uma calculadora usando a biblioteca Logos.
]{../codes/logos_example.rs}
\vspace{-1em}
\fonte{Adaptado de \citeonline{logos}.}

Além da simplicidade na definição dos \textit{tokens}, o analisador léxico gerado é extremamente rápido, como mostra o teste de desempenho do repositório oficial da biblioteca na \autoref{tab:logos_benchmark}.

\begin{table}
	\centering
	\caption{Teste de desempenho da biblioteca Logos.}
	{
		\begin{tabular}{ll}
			\hline
			\textbf{Teste}                          & \textbf{Benchmark}                 \\ \hline
			Identificadores                         & 647 ns/iter (+/- 27) = 1204 MB/s   \\
			Palavras-chave, operadores e pontuações & 2,054 ns/iter (+/- 78) = 1037 MB/s \\
			Strings                                 & 553 ns/iter (+/- 34) = 1575 MB/s   \\ \hline
		\end{tabular}
	}
	\fonte{Adaptado de \citeonline{logos}.}
	\label{tab:logos_benchmark}
\end{table}

Por fim, o uso da biblioteca Logos estará na implementação de toda a análise léxica, evitando que tempo seja gasto na análise manual de cada \textit{token}. A motivação para a escolha da biblioteca se deve à sua simplicidade e maturidade no ecossistema Rust, assim minimizando o tempo de desenvolvimento e garantindo maior estabilidade.
