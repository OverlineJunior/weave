\chapter{Introdução}\label{ch:introducao}

Nos últimos anos, as empresas têm adotado novas abordagens para o desenvolvimento de software, e dentre elas, a que se destaca é a metodologia ágil \cite{17thstateofagile}. Sabendo que tais metodologias favorecem a flexibilidade ao invés de planejamento rígido \cite{agile}, vem a necessidade de um modelo arquitetural que permite uma adaptação rápida a novas demandas. Um dos modelos que atendem esses requisitos é o padrão arquitetural \textit{Entity Component System} (ECS).

O padrão ECS surgiu na área de desenvolvimento de jogos, com um dos fatores sendo a alta necessidade de adaptação rápida na indústria. Por mais que ECS continue sendo majoritariamente aplicado em jogos, sua utilidade expande para qualquer aplicação que dependa fortemente de rápida iteração de desenvolvimento, flexibilidade ou desempenho \cite{flightdynamics}.

Frequentemente, o padrão ECS é abstraído em forma de biblioteca, em uma determinada linguagem de programação. Isso é devido ao fato de que a implementação do padrão, principalmente de forma eficiente, é composta de vários detalhes técnicos, como a organização dos dados na memória (Mertens, 2024). Muitas vezes, tais detalhes são irrelevantes para o desenvolvedor, e por isso eles são ocultados pela interface da biblioteca.

Dado isso, o projeto visa abstrair o padrão ECS através de um método diferente — um protótipo para uma linguagem de programação orientada a ECS. Será abordado desde as escolhas de \textit{design} da linguagem até a implementação do interpretador, por fim examinando, de forma qualitativa, se foi possível alcançar os objetivos determinados.

\section{Pergunta Problema}

Como projetar e desenvolver um protótipo de interpretador para uma linguagem de programação orientada a ECS, aproveitando suas vantagens nativas para uma melhor abstração do padrão?

\section{Objetivos}

\subsection{Objetivo Geral}

Investigar como o \textit{design} e a implementação de um protótipo de interpretador para uma linguagem de programação orientada ao ECS podem ser efetuados.

\subsection{Objetivos Específicos}

\begin{itemize}
    \item Definir os requisitos e princípios de \textit{design} da linguagem;
    \item Implementar um protótipo de interpretador funcional;
    \item Avaliar o impacto e a viabilidade do protótipo.
\end{itemize}

\section{Justificativa}

Sabendo que o padrão ECS se adequa as metodologias ágeis, que estão sendo amplamente adotadas pelas empresas \cite{17thstateofagile}, este trabalho propõe explorar o ECS de uma forma menos usual, mas que pode trazer benefícios para o padrão e para o desenvolvimento de software em geral.

A criação de uma linguagem de programação orientada ao ECS tem a capacidade de inovar na forma como o padrão é criado e utilizado, pois linguagens de programação permitem abstrações mais expressivas e específicas do que bibliotecas.

A lacuna se dá pelo fato da escassez de linguagens de programação que são projetadas especificamente para o padrão ECS, com a maioria das implementações sendo feitas em forma de bibliotecas para linguagens já existentes. Exceções incluem linguagens como \textit{Flecs Script} \cite{flecsscript} e \textit{Ecsact} \cite{ecsact}, porém, \textit{Flecs Script} é uma linguagem diretamente ligada a biblioteca \textit{Flecs} e \textit{Ecsact} é uma linguagem incompleta, dependendo de linguagens de programação existentes, como C\# \cite{ecsact}.
